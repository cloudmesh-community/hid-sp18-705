% status: 80
% chapter: TBD

\def\paperstatus{80} % a number from 0-100 indicating your status. 100
                % means completed
\def\paperchapter{TBD} % This section is typically a single keyword. from
                   % a small list. Consult with theinstructors about
                   % yours. They typically fill it out once your first
                   % text has been reviewed.
\def\hid{hid-sp18-705} % all hids of the authors of this
                                % paper. The paper must only be in one
                                % authors directory and all other
                                % authors contribute to it in that
                                % directory. That authors hid must be
                                % listed first
\def\volume{9} % the volume of the proceedings in which this paper is to
           % be included

\def\locator{\hid, Volume: \volume, Chapter: \paperchapter, 
	Status: \paperstatus. \newline}

\title{New Approaches to Managing Metadata at Scale in Research Libraries}
\author{Timothy A. Thompson}
\affiliation{%
  \institution{Indiana University Bloomington}
  \streetaddress{School of Informatics, Computing, and Engineering}
  \city{Bloomington} 
  \state{Indiana} 
  \postcode{47408}
}
\email{timathom@indiana.edu}

\begin{abstract} 
The analysis of big data often relies on distributed storage and
computation; however, access to big data---and to the platforms capable of
managing and processing it---continues to be largely centralized.
Centralization is particularly evident in the case of the metadata produced,
managed, and disseminated by academic and research libraries. Libraries
typically create and share their catalog records by uploading them to a
centrally managed database, which can then be searched by other libraries for
records that can be copied and added to an institution's local catalog. This
centralized approach, which operates on the basis of membership fees, has the
advantage of scalability and availability, but it comes at the cost of a loss
of autonomy. Although technical innovation is possible within the current
paradigm, the growing maturity of peer-to-peer protocols and decentralized
solutions points toward an alternative approach, one that would allow
libraries to share their data directly without having to pay an expensive
intermediary.
\end{abstract}

\keywords{i523, hid-sp18-705, Research Libraries, Library Catalogs,
	Blockchain, BigchainDB}

\maketitle

\section{Introduction}
The problem of entity resolution (also known as record linkage or data
matching~\cite{pC12}) is one that has a direct impact on the work of
information professionals in research libraries. In library units
responsible for catalog management, many workflows center on a procedure
known as copy cataloging, which aims to expedite the processing of new
acquisitions. Copy cataloging involves searching a shared database for
records created by another cataloging agency, but that describe identical
publications that have been acquired by one's local institution~\cite{cD17}.
In the current environment, a single company, the Online Computer Library
Center (OCLC---\url{http://www.oclc.org}), is the only viable platform for
global cooperative cataloging~\cite{aT10}. OCLC provides data aggregation
and warehousing services that allow libraries to effectively share their
data, but its business model does not encourage peer-to-peer interaction and
innovation among individual libraries. This vendor-driven paradigm entails
the acceptance of a business model that, in effect, charges libraries for
serving their own data back to them, with some added value through quality
control and normalization. Once a library's data has been sent to OCLC, it
also becomes subject to potential licensing restrictions, as well as the
expectation that future dissemination of the data will include attribution
of OCLC~\cite{oclcND, oclc10}.

\section{New Approaches to Metadata Management}
Libraries have a tradition of experience with record matching and
automation~\cite{jM92}, but now stand to benefit from the increasingly
mainstream availability of algorithms and routines developed within the
context of data science and machine learning. Sophisticated algorithms for
string comparison and probabilistic record linkage have long been
available, but are not widely used by libraries, with the exception of
large-scale projects such as the Social Networks and Archival Context
Project (SNAC) (\url{http://snaccooperative.org/}) and the Virtual
International Authority File (VIAF) (\url{http://viaf.org/}). The former has
employed methods based on Naive Bayes classification algorithms to aggregate
and disambiguate data from across a wide range of libraries and archives
(the reported accuracy of the approach fell with the range of 80-90
percent)~\cite{rL11}. More recent approaches to record matching have
improved on probabilistic methods such as Naive Bayes by using Artificial
Neural Networks, improving accuracy rates in some cases to 98 percent or
more~\cite{rG17}.

As machine learning tools and methods have become more accessible,
however, large-scale, real-time access to library metadata has not
necessarily followed suit. The catalog of a large academic library may
contain around 10 million records~\cite{yul18}. By comparison, as of August
2018, the OCLC catalog database, WorldCat, contained 427,501,671
bibliographic records in 491 languages~\cite{oclc18}. As long as service
providers such as OCLC maintain centralized control over the aggregated
metadata of research libraries, large-scale computational analysis---and the
innovation it could produce---will remain proprietary and locked away.

The situation is further complicated by professional and cultural norms
within libraries. Although decentralization may be appealing as an ideal,
librarians who manage bibliographic metadata are also immersed in a
discourse that centers on the idea of control: they use terms such as
authority control, controlled vocabularies, and intellectual and physical
control of collections~\cite{olson01}. The idea of control is closely
related to the idea of trust: when workflows and systems are centralized, it
becomes easier to enforce norms and standards, but it also becomes more
likely that potential contributors may be excluded, especially when they are
unable to afford the price of membership in a proprietary system.

New distributed technologies and protocols, including blockchains and
distributed hash tables (DHTs), could allow research libraries to form
robust peer-to-peer networks that would enable data sharing on a larger
scale. Although public blockchains such as Ethereum and Bitcoin are limited
in the amount of data that can feasibly be stored on chain, alternative
platforms that address this limitation have recently emerged. The
blockchain-based database service BigchainDB, implemented in Python,
provides a robust storage data solution while preserving the benefits of
blockchain features such as data immutability and an asset-based
transactional model. By running a consortium blockchain network of
BigchainDB nodes~\cite{vButerin15}, libraries could be empowered to abandon 
centralized models and begin managing their data collectively.

\section{Blockchains for Research Libraries}
Some in the library profession have been skeptical of blockchain
applications for their domain, arguing that they have been overhyped as a
panacea, when in reality they are nothing more than slow, expensive,
append-only databases~\cite{sjsu18}. Even core developers working to support
the Bitcoin blockchain have argued that the constraints imposed by
blockchain technology, such as immutability and decentralized consensus,
make it appropriate for a very limited set of applications---namely,
currency and the exchange of value~\cite{jSong18}. For individuals and
organizations who are investigating blockchains as a technical solution, it
is important from the outset to establish a framework for evaluating their
applicability and appropriateness~\cite{bS28}. For example, a
blockchain-based solution may be appropriate in a scenario in which there is
a lack of trust among participants, or in which processes and collaboration
would be more efficient if the need for trust were eliminated~\cite{bS28}.
In the case of a shared catalog for research libraries, trust is an issue
because not all participants can be trusted to provide data that conforms to
expected levels of quality. A commercial, centralized solution mitigates
these concerns by requiring participants to pay a membership fee. A
blockchain solution addresses issues of trust by enforcing a decentralized
consensus mechanism, which may take different forms, but which is designed
to ensure that participants can trust the network to maintain a consistent
state across all transactions~\cite{buchman2018latest}.

The Proof-of-Stake consensus algorithm, employed by some blockchain
networks as an alternative to Bitcoin's resource-intensive Proof-of-Work
mechanism, is similar to the membership fee model in that validator nodes
are elected based on their share of ``stake'' in the network, measured by
their willingness to commit or stake an allocation of network tokens as a
proof of honesty~\cite{gMarin18}. For research library applications, a
variation of Proof-of-Stake known as Proof-of-Authority may be the most
appropiate solution~\cite{gMarin18, vButerin15}. In contrast to public
blockchains such as Ethereum and Bitcoin, or fully private blockchains
restricted to a single organization, so-called consortium blockchains may be
the preferred approach, one in which consensus ``is controlled by a
pre-selected set of nodes''~\cite{vButerin15}. The model implemented by the
BigchainDB project fits the parameters of a consortium blockchain that
implements a Proof-of-Authority approach to consensus~\cite{bdb18b}.

\section{Design Requirements}
A blockchain-based catalog for research libraries should support the
creation of a decentralized marketplace for library metadata. Rather than
paying a centralized exchange to distribute their catalog records, libraries
could buy and sell records in a peer-to-peer exchange. Catalog records could
thus become a source of revenue rather than a costly expenditure. Many
blockchain systems support the creation of so-called smart assets, or the
creation of tokens to represent real-word assets. A new token could be
minted to facilitate the exchange of metadata objects, and payment and
settlement channels could be created using smart contracts on a public
blockchain such as Ethereum. However, a public blockchain solution does not
fully satisfy the requirements of decentralization for this use case. A data
asset cannot be represented exclusively by a token---it also needs to be
stored in a decentralized system optimized for read and write transactions.
Public blockchains such as Ethereum have been designed for exchange, not
storage. At the current price of the Ethereum blockchain's native token,
Ether (ETH), at approximately \$200.00, storing 1 Gigabyte of data on the
blockchain would cost over \$7,000,000.00~\cite{tHess16}. A decentralized
system for library metadata must be able to scale and store big data out of
the box. BigchainDB is a production-ready solution that meets the
requirements for this use case: it supports the creation of assets and the
direct storage of metadata objects on its blockchain~\cite{bdb18c}.

\section{Project Scope}
The current project presents findings from an exploration of BigchainDB as a 
blockchain database solution for a shared library catalog. It includes a 
preliminary analysis of library metadata requirements and whether they can be 
satisified using BigchainDB.

\section{BigchainDB}
\subsection{Evolution}
BigchainDB was created to address the scalability and storage
limitations of traditional blockchains such as Bitcoin and Ethereum and to
create a hybrid solution that builds a blockchain layer on top of an existing
big data system~\cite{bigDB18}. Development of the BigchainDB framework
initially focused on integration with the RethinkDB system, but now works
exclusively with MongoDB~\cite{ks16, bigDB18}.

The early focus of BigchainDB development was to create an
architecture that would allow existing big data databases to be
``blockchainified''~\cite{bigDB16a}. The original BigchainDB whitepaper,
released in June 2016, focused on the scalability limitations of traditional
blockchain networks such as Bitcoin and claimed that it should be possible
to develop a blockchain-based distributed database that would enable ``1
million writes per second throughput, storing petabytes of data, and
sub-second latency''---in contrast to the storage restrictions and 7
transaction-per-second (tps) limit of the Bitcoin network~\cite{bigDB16a}.
The advantages of adding a blockchain layer to an existing distributed
database would be to incorporate ``decentralized control, immutability, and
creation [and] movement of digital assets''~\cite{bigDB16a}. 

The primary challenge in designing a decentralized system is how to
defend against both arbitrary failure and malicious actors. In so-called
Sybil attacks, an attacker attempts to generate false identities in order to
gain majority control over a network~\cite{dJ02}. To address Sybil attacks,
BigchainDB proposes a governance model that would create a federation of
trusted nodes. Because all participants are known, any attempt by one
participant to gain control over the network would be obvious. A more
pervasive vulnerability comes in the form of the so-called Byzantine
Generals' Problem~\cite{bigDB16a}. Nodes in a distributed network must be
able to reach consensus about the final order of transactions at each state
of the system, even in the presence of node failure or malicious attempts to
manipulate system state in order to gain an unfair advantage---for example,
in double-spending, in which a transaction is replayed so that the same
asset can be used again (a particular problem in the case of financial
transactions)~\cite{bigDB16a, aA17}.

In its original design, BigchainDB relied on the consensus algorithm
of its underlying database to manage benign node failure and incorporated
additional constraints to verify the integrity of the voting process by which
nodes in the network approved transactions---and the blocks containing
them---as valid~\cite{bigDB16a}. However, in its initial version, BigchainDB
did not claim to be Byzantine Fault Tolerant (or BFT---the term used to
indicate that a system can withstand unexpected node behavior, whether benign
or malicious, up to a certain threshold~\cite{bigDB16a}). In the original
design, all nodes belonged to a single logical database. This made the system
overly centralized and vulnerable to attack: a malicious actor who gained
control over a single node would have been able to drop the entire database,
which was shared among all nodes in the network~\cite{ks16, bigDB18}.
BigchainDB 2.0, released in June 2018, underwent a complete redesign and
incorporated full Byzantine fault tolerance through integration with
Tendermint, an application for managing consensus and state machine
replication in blockchain systems~\cite{troyM18b, tender18}. As a result of
implementing Byzantine fault tolerance through Tendermint, BigchainDB's
original goal of supporting 1 million tps was no longer viable. 

\subsection{Benchmark}
A recent benchmark of BigchainDB 2.0 throughput performed by the
BigchainDB development team indicated that the system was able to process
approximately 300 tps~\cite{troyM18a, bdb18g}. Benchmarking was carried out on
a four-node network on Microsoft Azure-hosted virtual machines located in the
same data center. Three separate experiments were run to test different
options, and a full report of configurations and results is available for
review~\cite{bdb18g}. The primary experiment tested how long it would take to
commit 1 million transactions of 765 bytes each to the BigchainDB blockchain
under default settings. Results showed an average rate of 299.0 tps and a
median rate of 309.0 tps. All 1 million transactions were finalized in 56
minutes with no failures~\cite{bdb18g}.

\subsection{Architecture}
The architecture of a BigchainDB 2.0 network is shown in
Figure~\ref{f:bdb}, created by the BigchainDB development team. Each node
in the network is self-contained and includes its own MongoDB database and
Tendermint application server. Tendermint is used to manage consensus,
communication, and state replication among nodes, whereas the software that
is unique to BigchainDB is responsible for ``registering and tracking the
ownership of `assets'\thinspace''~\cite{troyM18b}. In BigchainDB 2.0, as is
the case in general with systems that are Byzantine Fault Tolerant, $3f + 1$
nodes are necessary to run a network, where $f$ is the number of faulty
nodes to be tolerated~\cite{bdb18}. Therefore, at least four nodes are
required in order to run a BigchainDB network: if one of the four nodes
becomes unresponsive or attempts to approve an invalid transaction, the
network will continue to function based on the majority consensus of the
other three nodes~\cite{bdb18}.

A BigchainDB client can potentially connect to any node in the network.
Each MongoDB instance contains a full replication of the data stored in the
network~\cite{bdb18}. The BigchainDB project officially supports three
client drivers to connect to a node server (in Python, Node.js, and
Java)~\cite{bdb18f}.

\subsubsection{BigchainDB Server}
The BigchainDB Server, written in Python, implements the logic to
model, validate, and store transactions in the BigchainDB
blockchain~\cite{troyM18b}. The server also incorporates a Python
implementation of the Crypto-Conditions specification, which is a standard
for enforcing complex boolean conditions for fulfillment (asset transfer)
using cryptographic signatures~\cite{cryptocon}.

All objects in BigchainDB are modeled as \emph{assets}. Two transaction types 
are available for managing assets: CREATE and TRANSFER~\cite{troyM18c}. Each 
transaction must be cryptographically signed with the private key of its 
``owner'' (the agent who created an asset through a CREATE transaction or to 
whom an asset was assigned through a TRANSFER transaction). Public/private 
keypairs are implemented using the Edwards-curve Digital Signature Algorithm 
Ed25519~\cite{troyM18c}. A transaction is encoded using a dictionary or 
associative array that can be serialized as a JSON object. The BigchainDB 
Transactions Specification defines the structure and usage of a BigchainDB 
transaction object~\cite{troyM18c}. Figure~\ref{c:bdb3} shows the key/value 
pairs that all valid BigchainDB transactions must include. 

Conditions for fulfillment and asset transfer are defined in the values
of the ``inputs'' and ``outputs'' keys. An object representing the asset
itself is stored as the value of the ``asset'' key and cannot be modified
once an asset has been created and committed to a block in the BigchainDB
blockchain. The ``metadata'' key is used to store an arbitrary object that
records additional information about the asset or its state: in contrast to
the asset object, the metadata object \emph{can} be modified with each
TRANSFER transaction~\cite{troyM18c}.

\subsubsection{Tendermint}
Tendermint provides an application interface and BFT consensus
algorithm for replicating application state across the nodes in a
decentralized network~\cite{tender18}. Tendermint Core implements the
consensus algorithm, which ensures that all nodes agree on a single order
for transactions. Tendermint's Application Blockchain Interface (ABCI)
provides a language-agnostic interface for blockchain applications to use
when validating and processing transactions~\cite{tender18}.

Figure~\ref{f:bdb2} is a sequence diagram, created by the BigchainDB
development team, that illustrates the role of Tendermint in processing
BigchainDB transactions. After a client prepares and signs a transaction,
typically using a BigchainDB driver, the transaction is submitted to the
BigchainDB server for initial validation. The server then sends the
transaction to Tendermint, which includes it in a local memory pool.
Tendermint returns its own validation request to the server and, upon
confirmation, proposes a new block and begins a round of voting as part of
its consensus algorithm. Each node in the network votes on the order and
validity of transactions in the block, and if consensus is reached, the
block is committed to the application's blockchain~\cite{gautam18, bigDB18}.
BigchainDB stores a queryable copy of each block in MongoDB, while
Tendermint appends each block to its canonical blockchain, which is stored
in an internal LevelDB database and used for replicating transaction state
to network peers~\cite{tender18, bigDB18}.

\subsubsection{MongoDB}
MongoDB is an enterprise-grade NoSQL database optimized for storing
JSON objects as documents. It supports both high availability (replication)
and scalability (sharding)~\cite{mongo18}. Early verions of BigchainDB used
a single MongoDB replication set and were able to take advantage of these
core MongoDB features. In BigchainDB 2.0, because each node maintains a
separate MongoDB instance, replication and sharding are not supported out of
the box~\cite{troyM18b}. In order to enforce practical immutability and
decrease the likelihood of data tampering, the BigchainDB server limits
access to MongoDB and does not expose any interfaces for deleting or making
arbitary modifications to database documents. Although it is technically
possible for a system administrator to modify the MongoDB database
directly, each BigchainDB transaction is signed with a public/private
cryptographic keypair---thus any tampering would result in a modified
signature, which would be detectable by other nodes in the network and would
violate its social contract~\cite{bigDB18}.

BigchainDB does take advantage of MongoDB's query facility for
read-only queries. It exposes a simple search interface through its HTTP
API, but also allows node administrators to create custom indexes and
leverage the full range of MongoDB query functionality~\cite{bdb18d}.

\section{Dataset}
The dataset for this project is intentionally small and meant to test a
potential use case for BigchainDB as a library catalog application.
Currently, library catalog records are stored in a set of industry-specific
formats maintained by the Library of Congress: the MAchine Readable
Cataloging (MARC) formats for bibliographic and authority data (standardized
as ISO 2709 and ANSI/NISO Z39.2)~\cite{kF12, lcnetdev}. In recent years, the
Library of Congress has undertaken an effort to update library metadata
standards and adopt standards and formats maintained by the World Wide Web
Consortium (W3C)---specifically those related to linked data and the
Semantic Web, such as the core data model known as the Resource Description
Framework (RDF)~\cite{lcbf, rdf11}. A new domain-specific data model and
ontology for library metadata, expressed using the W3C's OWL standard for
semantic ontologies, is currently being developed and
implemented~\cite{owl2}. The data used here for testing with BigchainDB
follows this model, known as the Bibliographic Framework Initiative
(BIBFRAME)~\cite{lcbf}.

In the basic model proposed by BIBFRAME, descriptions of library
resources are divided into three entity types or classes: Work (the abstract
concept of the resource), Instance (the embodiment of a Work in a particular
publication), and Item (a physical copy of an Instance)~\cite{lcbf}. As an
example for this project, a catalog record from the Indiana University
Library catalog was chosen. This record describes the Lilly Library's
partial copy of the Gutenberg Bible. The data is divided into six
files:

\begin{verbatim}
ocm05084045.xml
gutenberg-iul-item.rdf
gutenberg-iul-instance.json
gutenberg-iul-item.json
gutenberg-iul-record.json
gutenberg-work.json
\end{verbatim}

The file \verb|ocm05084045.xml| represents the original MARC-format
record, encoded as XML. The file \verb|gutenberg-iul-item.rdf| provides an
example of a partial conversion of the original MARC record to the BIBFRAME
model using the RDF/XML serialization~\cite{liam}. The remaining files
represent the data used to create assets for storage in BigchainDB and are
encoded in BIBFRAME using the JSON-LD serialization of RDF~\cite{liam}.
Several preprocessing steps of data conversion and cleanup were necessary.
The original MARC/XML catalog record was converted to BIBFRAME RDF/XML using
a suite of XSLT stylesheets provided by the Library of
Congress~\cite{lcnetdev2}. The RDF/XML documents were then converted to
JSON-LD using the Python RDFLib library (see the \verb|convert_rdf.py|
script in the \verb|project-code| directory for a brief example). The
JSON-LD produced by RDFLib was then broken into separate files to allow for
the creation of individual assets in BigchainDB.

\section{Implementation}
Currently, most large library catalogs are stored in enterprise
relational databases such as Oracle. The catalog is one module in a suite of
services known as an Integrated Library System (ILS), which also includes
modules for circulation and ordering or acquisitions. The cataloging module
in an ILS includes a public-facing interface for search and retrieval and a
staff-facing interface for data entry and management. One advantage of using
a distributed system such as BigchainDB for library cataloging functions
would be to allow libraries to share their data more easily with peer
institutions. BigchainDB's asset-based data model might also allow libraries
to perform inventory and lending functions more efficiently. However, many
functional components would need to be considered before determining whether
a blockchain platform such as BigchainDB would be appropriate for the
library catalog use case. This project focuses on one such component:
namely, the management of roles and permissions for data entry.

The Python component of this project implements an extension to
BigchainDB that adds support for Role-Based Access Control (RBAC)
functionality~\cite{gautam18b}. The code is based on a Node.js example
created by the BigchainDB development team to demonstrate the RBAC
extension~\cite{bdb18i}. Support for RBAC is important for library
cataloging because library personnel roles are typically divided between
professional librarians (catalogers) and paraprofessional technicians.
Librarians are expected to create ``original'' descriptions of library
resources, whereas paraprofessionals are responsible for copying existing
data from a shared database such as OCLC WorldCat. Public blockchain systems
do not usually impose write restrictions, so support for RBAC is an
important consideration when evaluating BigchainDB.

The Python script \verb|rbac_demo.py| connects to a BigchainDB server
instance and populates it with the BIBFRAME data described above. All
BigchainDB CREATE transactions must include a JSON-serializable object to
represent the asset being recorded on the blockchain. The \verb|asset| field
of a CREATE transaction takes an object with the required key \verb|data|.
The content of the \verb|asset| field is treated as immutable---it cannot be
changed once a CREATE transaction has been committed, or when ownership of
an asset is subsequently changed using a TRANSFER transaction.
Figure~\ref{c:bfw} shows how a Work asset might be represented in
BigchainDB. Because this data cannot be changed, it makes sense to represent
it simply using its RDF type (in this case, it is a BIBFRAME Work with a
subtype of Text), as well as a human-readable label. Any BigchainDB
transaction may also include an optional \verb|metadata| key that takes as
its value an arbitrary JSON object. This flexible design makes it possible
to effectively ``update'' data by using a TRANSFER transaction to indicate
that the state of an asset has changed---and recording that change in the
metadata object. The code in \verb|rbac_demo.py| creates separate BigchainDB assets
to represent the BIBFRAME types Work, Instance, and Item. The project data and code also illustrate 
how JSON-LD named graphs may be used to include
both descriptive and administrative metadata about the same BigchainDB asset
in a single transaction. 

\subsection{Role-Based Access Control in BigchainDB}
The file \verb|rbac.py| contains a single Python class, \verb|BigchainRbac()|, that provides an 
interface to create new assets, users, types, and type instances for Role-Based Access Control in 
BigchainDB. In the BigchainDB RBAC extension, roles and permissions, like everything else, are 
modeled as assets~\cite{gautam18b}. Two basic assets are necessary for bootstrapping: an 
asset to represent the application in which RBAC is being used and an asset to represent an admin 
group for admin users who can assign permissions. 

The BigchainDB impl


The \verb|gutenberg-work.json| file comprises two
named graphs: one representing the Work entity and one representing a
separate Record entity (which is not part of the core BIBFRAME model).
Within the named graph for the Record entity, there is an RDF property
(\verb|foaf:topic|) that links to the URI for the named graph representing
the Work entity. Figure~\ref{f:rbac} illustrates this pattern, indicating
how BigchainDB metadata objects may be used to create internal linkages
among assets conforming to the BIBFRAME data model.

\section{Conclusion}
Lorem ipsum dolor sit amet, consectetur adipiscing elit. Sed sodales
eleifend pharetra. Phasellus interdum augue nec sapien pretium accumsan.
Proin dapibus massa in enim pulvinar facilisis. Proin venenatis nisl metus,
nec tincidunt lorem viverra at. Quisque sagittis lectus a mi varius, a
auctor tortor dapibus. Nam magna ex, rutrum et mauris et, eleifend iaculis
enim. Sed non tortor quis ligula placerat lacinia. Proin consectetur, sapien
quis molestie volutpat, velit lacus faucibus quam, id rutrum dui est et
eros. Ut fermentum malesuada hendrerit. Lorem ipsum dolor sit amet,
consectetur adipiscing elit. Fusce tempus, ante at suscipit facilisis,
mauris urna auctor urna, eget pretium mi massa accumsan massa. Duis semper,
ex eu rhoncus elementum, est est tristique est, nec tristique orci nunc eget
nibh. In vestibulum purus at nibh egestas, eget convallis mi molestie.

Lorem ipsum dolor sit amet, consectetur adipiscing elit. Sed sodales
eleifend pharetra. Phasellus interdum augue nec sapien pretium accumsan.
Proin dapibus massa in enim pulvinar facilisis. Proin venenatis nisl metus,
nec tincidunt lorem viverra at. Quisque sagittis lectus a mi varius, a
auctor tortor dapibus. Nam magna ex, rutrum et mauris et, eleifend iaculis
enim. Sed non tortor quis ligula placerat lacinia. Proin consectetur, sapien
quis molestie volutpat, velit lacus faucibus quam, id rutrum dui est et
eros. Ut fermentum malesuada hendrerit. Lorem ipsum dolor sit amet,
consectetur adipiscing elit. Fusce tempus, ante at suscipit facilisis,
mauris urna auctor urna, eget pretium mi massa accumsan massa. Duis semper,
ex eu rhoncus elementum, est est tristique est, nec tristique orci nunc eget
nibh. In vestibulum purus at nibh egestas, eget convallis mi molestie.

Lorem ipsum dolor sit amet, consectetur adipiscing elit. Sed sodales
eleifend pharetra. Phasellus interdum augue nec sapien pretium accumsan.
Proin dapibus massa in enim pulvinar facilisis. Proin venenatis nisl metus,
nec tincidunt lorem viverra at. Quisque sagittis lectus a mi varius, a
auctor tortor dapibus. Nam magna ex, rutrum et mauris et, eleifend iaculis
enim. Sed non tortor quis ligula placerat lacinia. Proin consectetur, sapien
quis molestie volutpat, velit lacus faucibus quam, id rutrum dui est et
eros. Ut fermentum malesuada hendrerit. Lorem ipsum dolor sit amet,
consectetur adipiscing elit. Fusce tempus, ante at suscipit facilisis,
mauris urna auctor urna, eget pretium mi massa accumsan massa. Duis semper,
ex eu rhoncus elementum, est est tristique est, nec tristique orci nunc eget
nibh. In vestibulum purus at nibh egestas, eget convallis mi molestie.







Shown in Figure~\ref{f:rbac}.
Shown in Figure~\ref{f:rbac2}.

\begin{figure}[!htb]
	\centering\includegraphics[width=\columnwidth]{images/bdb-arch.pdf}  
	\caption{High-Level Architecture of BigchainDB 
		2.0~\cite{bdb18d}}\label{f:bdb}
\end{figure}

\begin{figure}[!htb]
	\centering\includegraphics[width=\columnwidth]{images/bdb-seq.pdf}  
	\caption{BigchainDB Sequence Diagram~\cite{gautam18}}\label{f:bdb2}
\end{figure}

\begin{figure}[!htb]
    \begin{verbatim}
{
  "id": ctnull,
  "version": version,
  "inputs": inputs,
  "outputs": outputs,
  "operation": operation,
  "asset": asset,
  "metadata": metadata
}	
    \end{verbatim}
    \caption{Required key/value pairs in a valid BigchainDB 
    transaction~\cite{troyM18c}}\label{c:bdb3}
\end{figure}

\begin{figure}[!htb]
	\begin{verbatim}
{
  "data": {
    "@context": {
      "rdfs": "http://www.w3.org/2000/01/rdf-schema#",
      "schema": "http://schema.org/"
    },
    "@type": [
      "http://id.loc.gov/ontologies/bibframe/Work",
      "http://id.loc.gov/ontologies/bibframe/Text"
    ],
    "rdfs:label": "Bible. Latin. Vulgate. 1454."
  }
}
	\end{verbatim}
	\caption{Representation of a Work asset in BigchainDB}\label{c:bfw}
\end{figure}


\begin{figure}[!htb]
    \begin{verbatim}
<rdf:RDF xmlns:bf="http://id.loc.gov/ontologies/bibframe/"
  xmlns:bflc="http://id.loc.gov/ontologies/bflc/"
  xmlns:madsrdf="http://www.loc.gov/mads/rdf/v1#"
  xmlns:rdf="http://www.w3.org/1999/02/22-rdf-syntax-ns#"
  xmlns:rdfs="http://www.w3.org/2000/01/rdf-schema#">
  <bf:Item rdf:about="http://example.org/ocm05084045#Item050-8">
    <bf:shelfMark>
      <bf:ShelfMarkLcc>
        <rdfs:label>BS75 1454</rdfs:label>
      </bf:ShelfMarkLcc>
    </bf:shelfMark>
    <bf:heldBy 
      rdf:resource="http://id.loc.gov/vocabulary/organizations/inuli"/>
  </bf:Item>
</rdf:RDF>
    \end{verbatim}
    \caption{Excerpt of Indiana University Library catalog record converted to 
        RDF/XML}\label{c:rdfxml}
\end{figure}

\begin{figure}[!htb]
    \begin{verbatim}
[  
  {    
    "@id": "http://example.org/ocm05084045#Item050-8",
    "@type": ["http://id.loc.gov/ontologies/bibframe/Item"],
    "http://id.loc.gov/ontologies/bibframe/heldBy": [{
      "@id": "http://id.loc.gov/vocabulary/organizations/inuli"
    }],
    "http://id.loc.gov/ontologies/bibframe/itemOf": [{
      "@id": "http://example.org/ocm05084045#Instance"
    }],
    "http://id.loc.gov/ontologies/bibframe/shelfMark": [{
      "@id": "_:Nb346817b2a894e1894d7c37a823ed84b"
    }]
  },
  {
    "@id": "_:Nb346817b2a894e1894d7c37a823ed84b",
    "@type": [
      "http://id.loc.gov/ontologies/bibframe/ShelfMarkLcc"
    ],
    "http://www.w3.org/2000/01/rdf-schema#label": [{
      "@value": "BS75 1454"
    }]
  }
]
    \end{verbatim}
    \caption{Excerpt of Indiana University Library catalog record converted 
        from RDF/XML to JSON-LD using the RDFLib library}\label{c:jsonld}
\end{figure}


\begin{figure}[!htb]
	\centering\includegraphics[width=\columnwidth]{images/assets-metadata.pdf}  
	\caption{Graph of asset and metadata objects in BigchainDB}\label{f:rbac}
\end{figure}

\begin{figure}[!htb]
	\centering\includegraphics[width=\columnwidth]{images/rbac-graph.pdf}  
	\caption{Graph of permissions in BigchainDB using Role-Based Access 
	Control}\label{f:rbac2}
\end{figure}




\begin{acks}
The author would like to thank Dr.~Gregor~von~Laszewski and the i523
and i516 teaching assistants for their support and suggestions in writing
this report.
\end{acks}

\bibliographystyle{ACM-Reference-Format}
\bibliography{report} 

